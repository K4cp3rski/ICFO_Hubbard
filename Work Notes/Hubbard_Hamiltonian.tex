\documentclass[12pt,a4paper]{article}
\usepackage[utf8]{inputenc}
\usepackage[T1]{fontenc}
\usepackage{fontspec}
\usepackage{amsmath}
\usepackage{graphicx}
\usepackage{mathtools}
\usepackage[table,xcdraw,dvipsnames]{xcolor}
\usepackage{hhline}
\usepackage{placeins}
\usepackage[margin=0.6in]{geometry}
\usepackage{colortbl}
\usepackage{physics}
\usepackage{float}
\usepackage{caption}
\usepackage{subcaption}
\usepackage{datetime}
\usepackage{hyperref}
\usepackage{amsfonts}
\usepackage{fancyhdr}
\usepackage{xparse}
\usepackage{tikz}
\usepackage[shortlabels]{enumitem}
\usetikzlibrary{calc}

\DeclareDocumentCommand{\hcancel}{mO{0pt}O{0pt}O{0pt}O{0pt}}{%
    \tikz[baseline=(tocancel.base)]{
        \node[inner sep=0pt,outer sep=0pt] (tocancel) {#1};
        \draw[red] ($(tocancel.south west)+(#2,#3)$) -- ($(tocancel.north east)+(#4,#5)$);
    }%
}%

\newcommand{\pd}[1]{\partial_{#1}}

\title{Hubbard Hamiltonian}
\author{Kacper Cybiński}
\newdate{date}{25}{08}{2022}
\date{\displaydate{date}}
% \date{\today}

\begin{document}

    \maketitle

    \begin{abstract}
        Following notes file is a detailed walkthrough of entry steps into many-body physics field I had at ICFO during my summer stay in the August of 2022. I outline here process of construction, and exact diagonaliation of Fermi- and Bose- Hubbard model, in variation of either one or two component bases, representing either spinful, or spinless systems.
    \end{abstract}

    \section{Basic Code Blocks}
    
\end{document}